\documentclass[conference]{IEEEtran}
\usepackage{blindtext, graphicx, subfiles}

% *** GRAPHICS RELATED PACKAGES ***
%
\ifCLASSINFOpdf
  % \usepackage[pdftex]{graphicx}
  % declare the path(s) where your graphic files are
  % \graphicspath{{../pdf/}{../jpeg/}}
  % and their extensions so you won't have to specify these with
  % every instance of \includegraphics
  % \DeclareGraphicsExtensions{.pdf,.jpeg,.png}
\else
  % or other class option (dvipsone, dvipdf, if not using dvips). graphicx
  % will default to the driver specified in the system graphics.cfg if no
  % driver is specified.
  % \usepackage[dvips]{graphicx}
  % declare the path(s) where your graphic files are
  % \graphicspath{{../eps/}}
  % and their extensions so you won't have to specify these with
  % every instance of \includegraphics
  % \DeclareGraphicsExtensions{.eps}
\fi

% correct bad hyphenation here
\hyphenation{op-tical net-works semi-conduc-tor}


\begin{document}
%
% paper title
% can use linebreaks \\ within to get better formatting as desired
\title{Improving the Security of Cloud Based Systems}


% author names and affiliations
% use a multiple column layout for up to three different
% affiliations
\author{\IEEEauthorblockN{Joseph LeRoy}
\IEEEauthorblockA{Graduate School of Arts and Sciences\\Computer and Information Science Department\\
Fordham University\\
Bronx, New York 10458\\}}



% use for special paper notices
%\IEEEspecialpapernotice{(Invited Paper)}




% make the title area
\maketitle


\begin{abstract}
\subfile{abstract}
\end{abstract}

\section{Introduction}
\subfile{Introduction/introduction}
\subsection{Cloud Computing Defined}
\subfile{Introduction/cloud-computing-defined}

\section{Hacking Groups Target Cloud Based Systems}
\subfile{Section2/hacking-groups}
\subsection{Intrusion Detections Defined}
\subfile{Section2/intrusion-detections-defined}
\subsection{Hypervisor Vulnerabilities}
\subfile{Section2/hypervisor-vulnerabilities}

\section{Proposed Countermeasures}
\subfile{Section3/proposed-countermeasures}
\subsection{Bro Intrusion Detection System}
\subfile{Section3/bro-ids}
\subsection{Tripwire Intrusion Detection System}
\subfile{Section3/tripwire-ids}
\subsection{Threat Intelligence}
\subfile{Section3/threat-intelligence}
\subsection{Log Aggregation, Correlation, and Analysis}
\subfile{Section3/log-aggregation-correlation-analysis}
\subsubsection{Elasticsearch}
\subfile{Section3/elasticsearch}
\subsubsection{Logstash}
\subfile{Section3/logstash}
\subsubsection{Kibana}
\subfile{Section3/kibana}
\subsubsection{Elastic Beats}
\subfile{Section3/elastic-beats}

\section{Conclusion}
\subfile{Conclusion/conclusion}

\appendices
\section{Proof of the First Zonklar Equation}
\blindtext

% use section* for acknowledgement
\section*{Acknowledgment}


The authors would like to thank...


% Can use something like this to put references on a page
% by themselves when using endfloat and the captionsoff option.
\ifCLASSOPTIONcaptionsoff
  \newpage
\fi



% trigger a \newpage just before the given reference
% number - used to balance the columns on the last page
% adjust value as needed - may need to be readjusted if
% the document is modified later
%\IEEEtriggeratref{8}
% The "triggered" command can be changed if desired:
%\IEEEtriggercmd{\enlargethispage{-5in}}

% references section

% can use a bibliography generated by BibTeX as a .bbl file
% BibTeX documentation can be easily obtained at:
% http://www.ctan.org/tex-archive/biblio/bibtex/contrib/doc/
% The IEEEtran BibTeX style support page is at:
% http://www.michaelshell.org/tex/ieeetran/bibtex/
%\bibliographystyle{IEEEtran}
% argument is your BibTeX string definitions and bibliography database(s)
%\bibliography{IEEEabrv,../bib/paper}
%
% <OR> manually copy in the resultant .bbl file
% set second argument of \begin to the number of references
% (used to reserve space for the reference number labels box)
\begin{thebibliography}{1}

\bibitem{IEEEhowto:kopka}
H.~Kopka and P.~W. Daly, \emph{A Guide to \LaTeX}, 3rd~ed.\hskip 1em plus
  0.5em minus 0.4em\relax Harlow, England: Addison-Wesley, 1999.

\end{thebibliography}

% biography section
%
% If you have an EPS/PDF photo (graphicx package needed) extra braces are
% needed around the contents of the optional argument to biography to prevent
% the LaTeX parser from getting confused when it sees the complicated
% \includegraphics command within an optional argument. (You could create
% your own custom macro containing the \includegraphics command to make things
% simpler here.)
%\begin{biography}[{\includegraphics[width=1in,height=1.25in,clip,keepaspectratio]{mshell}}]{Michael Shell}
% or if you just want to reserve a space for a photo:

\begin{IEEEbiography}[{\includegraphics[width=1in,height=1.25in,clip,keepaspectratio]{picture}}]{John Doe}
\blindtext
\end{IEEEbiography}

% You can push biographies down or up by placing
% a \vfill before or after them. The appropriate
% use of \vfill depends on what kind of text is
% on the last page and whether or not the columns
% are being equalized.

%\vfill

% Can be used to pull up biographies so that the bottom of the last one
% is flush with the other column.
%\enlargethispage{-5in}




% that's all folks
\end{document}
