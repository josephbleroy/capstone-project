\documentclass[./main.tex]{subfiles}

\begin{document}
Hypervisors have had their fair share of vulnerabilities in the past. The relevance of cloud based systems has made them an incredibly sought after target for ethical and malicious hackers. Due to the multitenancy nature of cloud based systems, arbitrary code can be executed to gain access to other virtual machines provisioned by the hypervisor. There are several types of vulnerabilities that have been discovered in recent years. One of the most talked about vulnerabilities was the VENOM (Virtualized Environment Neglected Operations Manipulation) security vulnerability or CVE-2015-3456. If a hacker were to exploit the VENOM vulnerability they would be able to escape the virtual machine and laterally move throughout the resource pool, and potentially move to other resource pools outside of its origin.

The Xen hypervisor technology employed by Amazon Elastic Cloud (EC2) has experienced about nine different vulnerabilities this year alone, as of May 20th. The most recent vulnerabilities affected the availability of services powered by the Xen hypervisor, as well as privilege sabotage and the ability to gain information through unauthorized channels.

In figure 3, an example of the VENOM (Virtualized Environment Neglected Operations Manipulation) security vulnerability shows how one vulnerable virtual environment, originating from “Resource Pool A” is able to escape to the hypervisor and then proceed to “Resource Pool B” and move to other virtual machines outside of its originating resource pool. This vulnerability is comparable to a quick spreading fire within an apartment complex without fire walls and partitions which prevent the expansion of damage to other apartments in the vicinity.

There are also other types of vulnerabilities common to cloud based systems that pose a huge security risk to both customers and providers of the service. Such being the case in side channel attacks. Hypervisor exploits that avoid normal segregation policies allow other virtual machines to collect artifact data left behind by virtual machines using shared hardware channels, such as the CPU cache. A recent exploitation coined “CacheBleed” “exploits information leaks through cache-bank conflicts in Intel processors”. A successful attack can “recover 2048-bit and 4096-bit RSA secret keys from OpenSSL 1.0.2f running on Intel Sandy Bridge processors after observing only 16,000 secret-key operations” (https://ssrg.nicta.com.au/projects/TS/cachebleed/).
\end{document}
