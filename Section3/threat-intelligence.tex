\documentclass[./main.tex]{subfiles}

\begin{document}
Threat intelligence is comprised of many different disciplines which in turn make it an entire field within the realm of information security. SANS describes CTI (Cyber Threat Intelligence) through a five stage process: victim, delivery, infrastructure, motivation, and actors. The FBI defines intelligence as any “information that has been analyzed and refined so that it is useful to policymakers in making decisions —specifically, decisions about potential threats to national security” (SANS CTI).

The difference between information and intelligence is key to understand the importance of threat intelligence. Information is simply the log file files and events that occur on a network or within a system. Intelligence is the processed, analyzed, and qualified data from log files and events. By itself, Bro has many powerful threat detecting and vulnerability mitigating features. However, there are ways to make Bro and even more resilient security product. Critical Stack, a security company owned by Capital One, developed an intelligence marketplace to enhance the security event detection rate of Bro. The feeds are populated with malicious indicators, such as Tor exit nodes, malicious IP addresses, malware domain lists, C3 (command and control) systems, among others. There are over 1.3 million malicious indicators and over 100 intelligence feeds and they’re easily digested into Bro’s intrusion detection capability.

It’s important to note that threat intelligence always requires human intervention as well as a proper and sustainable business plan to ensure security is appropriately addressed, from the executive branch down. The tools and log analysis features that Bro provides allows us to gain intelligence on the malicious actors that target our servers. From there intelligence practices can be implemented to improve network resiliency. The action of examining log files, populating threat feeds, and interpreting events helps build reports such as the Verizon DBIR (Data Breach Investigation Report) and IBM’s Cost of Data Breach Study.
\end{document}
