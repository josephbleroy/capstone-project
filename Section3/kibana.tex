\documentclass[./main.tex]{subfiles}

\begin{document}
Kibana is a frontend application that displays a visual representation of post processed system log contents. It works by connecting to Elasticsearch and encrypting log data sent from the application servers using a public and private key pair. One of the powerful features to Kibana is its ability to tie into various departmental functions of a business. Kibana allows you to display log data in the following formats: area charts, data tables, line charts, markdown widgets, metrics, pie charts, tile maps, and vertical bar charts. One of the most interesting visualization features is displaying geographic IP addresses as latitude and longitudes. By translating the IP address to a geographic location, Kibana can display the geographic location of the host generating the log entry, associate it to an event. In return, this increases the situational awareness through deductive reasoning. If thousands of attacks originate from Russia, firewall rules to block or filter the traffic should be implemented and enabled.

Through extensive modification, an internal network can be geographically mapped by region, individual location, or department, thus allowing for a more robust endpoint protection system. This would greatly aid in an incident response situation to determine where a vulnerable system exists on the network, and from which part of the office it resides. The heart of Kibana is written using mostly JavaScript, and is completely open source on GitHub. It can be modified to fit the needs of any business and has many features which make it a great application to use for a SIEM (Security Information and Event Management) system. In a typical setup, Nginx will be used as a reverse proxy to allow external web access to Kibana, which connects to the entire ELK stack. This allows for more stringent security controls, such as basic authentication of HTTP users and enforcing SSL encryption. This should be sufficient for a non­production environment.
\end{document}
