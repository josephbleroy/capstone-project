\documentclass[./main.tex]{subfiles}

\begin{document}
Log data is one of the most important resources security teams have at their disposal. Whether security teams are responding to an event or gathering intelligence, the process of aggregating the variety of logs is a challenging task, especially on cloud based systems. Normalizing log data so that it can be implemented into a data store is another important task. Each log, whether it be authentication logs, boot logs, MySQL logs, or web server logs, have very different ways of being interpreted. Each log speaks their own language and cannot be digested universally. Associating correlated data with values based on their function is essential to being able to create a mutual connection between the contents and some identifiable, perhaps known malicious value. It is essential to analyze data from a high-level perspective, such as a graph or a dashboard with multiple graphs, to better understand who, what, and why in a security incident.

UNIX based operating systems have become primarily focused around a select variety of Linux kernel distributions. One of the most popular flavors of Linux is Ubuntu, which was released in October of 2004. In Ubuntu, logs are stored in the /var/log/ directory and typically contain a collection of authentication logs, boot logs, kernel logs, and if installed also contain web server logs. This helps system administrators troubleshoot issues that arise. It also provides the security community with breadcrumbs and artifacts to track hackers back to their origins and to understand how and what systems and applications were exploited during an incident.
\end{document}
