\documentclass[./main.tex]{subfiles}

\begin{document}
Elasticsearch is a “schema free, RESTful (Representational State Transfer) and JSON (JavaScript Object Notation) based distributed document store” that requires close to zero configuration for a simple deployment. It’s used by companies such as GitHub, Mozilla, SoundCloud, and Stack Overflow, just to name a few. The core of Elasticsearch is powered by the Apache Lucene library. Lucene allows Elasticsearch to gracefully process full text search and indexing of large amounts of data.

One of the many powerful features of Elasticsearch is that it is easy to setup a node cluster, which ensures reliability and the overall health of the data store, which contain important log files and other security information. In a production level environment at a SOC (Security Operations Center) it would be imperative to ensure data is always available and in near real-time. You can accomplish this in multiple ways. The first way is through data duplication and the other is through data partitioning. One of the Amazon AWS Elastic Cloud (EC2) clusters that I configured using the ELK stack uses a data duplication method. Through this process the same exact log data is split between two nodes, communicated between each other to ensure they both have the same data, and the load is split between each other using the round robin scheduling algorithm by implementing an ELB (Elastic Load Balancer). This is just a traffic load balancer for incoming web requests. Another way to distribute data using Elasticsearch is by using a data partitioning method. This simply allows the node requesting data to search over a wide number of data stores rather than just one. A data partitioning setup is typically implemented when there’s a huge amount of data being stored, which surpasses the requirements of any one system.

Elasticsearch is very easy to manipulate and the core of its power is accomplished through a plugin based ecosystem. Since Java is required, custom JAR files can be developed as plugins to increase the functionality of Elasticsearch. Site plugins can also be developed to increase functionality and require some knowledge of HTML, CSS, and JavaScript. Additionally, a mix between Java and site plugins can also be developed. The only difference between the two main categories is that Java plugins are required by all nodes running Elasticsearch, whereas site plugins can exist independently.
\end{document}
