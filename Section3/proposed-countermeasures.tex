\documentclass[./main.tex]{subfiles}

\begin{document}
There are many possible ways to improve the security of cloud based systems. In recent years the concept of Linux Containers (LXC) and software such as Docker Engine have come to light as a possible replacement for virtual machine environments, which currently exist on nearly a half-million Amazon Elastic Cloud (EC2) servers throughout the world. Eliminating the virtualization process allows the host operating system to reduce the “overhead that comes with running a separate kernel and simulating all the hardware” (https://linuxcontainers.org/). Linux Containers employ many security enhancements over virtualized machines by removing the hardware layer and thus reducing the attack range for hackers to exploit. Several security features from the Linux Kernel are used in Linux Container technology, such as: Namespaces, AppArmor (Application Armor), Seccomp, Chroots, Kernel Capabilities, and Control Group functionality (cgroups). The preceding list of security features is non-extensive and will continue to grow as Linux Container technology matures.

Another way to improve the security profile of cloud based systems is to integrate the Docker Engine into a currently existing environment. Docker containers, which run on top of Docker Engine, are lightweight by definition and allow developers to provide continuous integration and delivery. There are many ways that Docker can improve security on virtual machines, such as building a security centric container that contains a firewall, an intrusion detection system, and a log collection and visualization software stack. Docker can operate in both a stateful or stateless manner, but excels running stateless applications. For example, a container can be created to encapsulate an Nginx reverse proxy or an Elasticsearch powered database for storing log files.

An ideal environment presented in figure 6 could be implemented into an existing cloud based environment to protect application servers. Using some of the software and other technology listed in this paper, such as Bro and the ELK stack, the security profile can be greatly enhanced and thus protect not only the residing resource pool contents but potentially other data pools within the datacenter. Further research will explore an even greater approach to enhancing the security on cloud based systems by using Docker to develop containers for each security mechanism, thus inheriting not only resource saving benefits but also Linux Container security enhancements.
\end{document}
