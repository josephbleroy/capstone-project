\documentclass[./main.tex]{subfiles}

\begin{document}
 The expansion of open source software communities has accelerated greatly over the past decade with the emergence of cloud based systems, the growth and speed of networks, device storage and speed improvements, among other advancements in technology. A cloud based provider allows its customers to rapidly provision virtual server instances, with a generous share of server resources, and within any geographic location. This process is orchestrated using open source hypervisors such as Xen, used by Amazon Web Services, and Kernel-based Virtual Machine (KVM), used by Digital Ocean. There are also many closed source hypervisors such as zVM, developed by IBM, and ESXi, by VMWare. However, both of these hypervisors have an incredibly complex source code which allows for vulnerabilities and other malicious artifacts to degrade the security profile of cloud based systems as a whole.

 Another area of concern within cloud based systems is the integrity of the security profile due to the multi-tenancy, or sharing of resources and services, within each physical system. Each time a new virtual server is created an IP address, resources, and host name are assigned to it. There are two types of hypervisors that create and run virtual machines, listed in figures 1 and 2. The creation process begins at the physical hardware, moves through the host operating system and hypervisor (KVM) or solely the hypervisor (Xen), and then a virtual machine is created and logically separated from other hosts on the network. This new host coexists among many other hosts, some of which have no security controls in place. This presents a unique security concern due to the intrinsic value of the security profile being degraded by potential vulnerable software on other hosts that share the same resource pool. However, the cost savings provided through “carpooling” server resources has created an illusion of confidentiality and integrity, thus enabling malicious actors to exploit vulnerabilities on security negligent systems.
\end{document}
