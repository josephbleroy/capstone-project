\documentclass[./main.tex]{subfiles}

\begin{document}
 Amazon Web Services has defined cloud computing as “the on-demand delivery of IT resources and applications via the Internet with pay-as-you-go pricing” (https://aws.amazon.com/what-is-cloud-computing/). Cloud computing is basically a system for automating processes through self-service virtualized environments. This model enables individuals and businesses to rapidly deploy virtualized servers within a matter of seconds without incurring large costs when compared to traditional hosting. In many ways the difference is similar to renting an apartment versus purchasing a home.
There are several characteristics that make cloud computing unique when compared to traditional datacenters. First, the systems that make up the cloud datacenter are required to be easily provisioned without aid from the hosting company. This process should be available during any hour of the day. Second, systems on the cloud hosting provider’s network should be available to all major cities and geographic locations. Next, the amount of storage, consumed or allotted bandwidth, CPU and RAM should be measured against an hourly pricing model to generate a variable monthly consumption bill. Finally, the hypervisor should allocate storage, CPU and RAM by resource pooling. A resource pool is a “logical abstraction for flexible management of resources” (vSphere 4.1 – ESX and vCenter). Based on the resource pool’s consumption and the total amount of resources the hypervisor has to allocate, it divides the system into multiple virtualized systems and scales them accordingly. For example, if the hypervisor host has a 6 GHz clock frequency and 12 GB of RAM, you could have three resource pools with a 2 GHz clock frequency and 4 GB of RAM.

Cloud computing is also comprised of two different models. There are service models, which include Infrastructure as a Service (IaaS), Platform as a Service (PaaS), and Software as a Service (SaaS). An IaaS provides the virtual machines, servers, storage, load balancers, and network components. A PaaS includes items such as databases and development tools, which are Service-Oriented Architectures, such as IBM Bluemix, Amazon Elastic Beanstalk, and Heroku. There’s also different deployment models, which define the type of access and controls each Cloud network contains. There are private, hybrid, community, and public Cloud deployment models. Community based Clouds are certified to host websites that are required to adhere to regulatory compliance laws. Examples are Payment Card Industry Data Security Standard (PCI DSS), Health Insurance Portability and Accountability Act (HIPAA), the Federal Information Security Management Act (FISMA), among others.

Nearly fifteen years ago Amazon Web Services introduced the concept of cloud computing and since then it has become the de facto standard for hosting digital content. Initially, the idea was to host virtualized systems for Amazon’s ecommerce website. It took them four more years to release the Simple Storage Solution (S3) and the “pay-as-you-go” model. Several months later, Amazon Elastic Cloud (EC2) was released, allowing businesses to use them as their hosting provider. In 2008, Google App Engine was released as a Platform as a Service, and a year later Microsoft launched Azure Beta. Cloud computing finally caught on as a mainstream component for businesses to rapidly deploy and scale infrastructures, platforms, and software depending on their budget and production requirements.

\end{document}
